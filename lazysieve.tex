\documentclass{article}
\usepackage[utf8]{inputenc}
\usepackage{amsfonts}
\usepackage{amssymb}
\usepackage{amsthm}
\usepackage{color}
\usepackage{mathtools}

\title{Lusta szita}
\date{2018.4.13.}

\newcommand*{\bigOrdo}{\ensuremath{\mathcal{O}}}
\newcommand*{\Integers}{\ensuremath{\mathbb{Z}}}
\newcommand*{\NaturalsP}{\ensuremath{\mathbb{N}^+}}
\newcommand*{\NaturalsZ}{\ensuremath{\mathbb{N}_0}}
\newcommand*{\Trees}{\ensuremath{\mathcal{T}}}

\newcommand*{\fFix}{\ensuremath{\operatorname{fix}}}
\newcommand*{\fInv}{\ensuremath{\operatorname{Inv}}}
\newcommand*{\fMove}{\ensuremath{\operatorname{move}}}

\newcommand*{\sAccumulated}[1]{\ensuremath{\operatorname{a}\left(#1\right)}}
\newcommand*{\sChildren}[1]{\ensuremath{\operatorname{c}\left(#1\right)}}
\newcommand*{\sDiff}[1]{\ensuremath{\operatorname{d}\left(#1\right)}}
\newcommand*{\sHeight}[1]{\ensuremath{\operatorname{h}\left(#1\right)}}
\newcommand*{\sLeft}[1]{\ensuremath{\operatorname{l}\left(#1\right)}}
\newcommand*{\sPosition}[1]{\ensuremath{\operatorname{q}\left(#1\right)}}
\newcommand*{\sPrime}[1]{\ensuremath{\operatorname{p}\left(#1\right)}}
\newcommand*{\sPrimes}[1]{\ensuremath{\operatorname{s}\left(#1\right)}}
\newcommand*{\sRight}[1]{\ensuremath{\operatorname{r}\left(#1\right)}}

\newcommand*{\vC}{\ensuremath{\operatorname{c}}}
\newcommand*{\vD}{\ensuremath{\operatorname{d}}}
\newcommand*{\vH}{\ensuremath{\operatorname{h}}}
\newcommand*{\vI}{\ensuremath{\operatorname{i}}}
\newcommand*{\vL}{\ensuremath{\operatorname{l}}}
\newcommand*{\vK}{\ensuremath{\operatorname{K}}}
\newcommand*{\vN}{\ensuremath{\operatorname{n}}}
\newcommand*{\vP}{\ensuremath{\operatorname{p}}}
\newcommand*{\vR}{\ensuremath{\operatorname{r}}}
\newcommand*{\vT}{\ensuremath{\operatorname{t}}}
\newcommand*{\vU}{\ensuremath{\operatorname{u}}}
\newcommand*{\vV}{\ensuremath{\operatorname{v}}}

\begin{document}

\small

\maketitle

\tableofcontents

\section{Bevezető}

\begin{align*}
\forall \vP: \forall \vI: \text{lehúz } \vI \vP
\end{align*}
helyett
\begin{align*}
\forall \vN: \text{lehúz \vN, ha } \exists \vP, \vI: \vN=\vI\vP
\end{align*}

\section{Fák}

A $ \Trees_0 $ a levelek halmaza, egy levél egy prím többszöröseinél szitál.
A $ \Trees_{\vN+1} $ a két, legfejlebb $\vN$  magas fából összevonással képzett fák halmaza. Egy elágazás ott szitál, ahol valamelyik gyereke szitál.

A szitaalgoritmus egy fát a szitatáblán számról számra léptet, és kiértékeli,
hogy a fa tartalmaz-e prímet, ami az aktuális számot szitálja.

\begin{align*}
\Trees_0 =&
	\{0\}
	\times \NaturalsZ
	\times \Integers
	\times \NaturalsP \\
\Trees_{\vH+1} =&
	\{\vH+1\}
	\times \NaturalsZ
	\times \Integers
	\times \NaturalsZ
	\times \bigcup_{\vI=0}^{\vH} \Trees_{\vI}
	\times \bigcup_{\vI=0}^{\vH} \Trees_{\vI}
	& (\vH \in \NaturalsZ) \\
\Trees =& \bigcup_{\vH=0}^{\infty} \Trees_{\vH}
\end{align*}

A szelektorfüggvények legyenek olyanok, hogy
\begin{align*}
\forall \vT \in \Trees_0: &\vT = \left(
	\sHeight{\vT},
	\sPosition{\vT},
	\sDiff{\vT},
	\sPrime{\vT}
	\right) \\
&\sChildren{\vT} = \emptyset \\
&\sPrimes{\vT} = \left\lbrace \sPrime{\vT} \right\rbrace \\
\forall \vH \in \NaturalsP, \vT \in \Trees_{\vH}: &\vT = \left(
	\sHeight{\vT},
	\sPosition{\vT},
	\sDiff{\vT},
	\sAccumulated{\vT},
	\sLeft{\vT},
	\sRight{\vT}
	\right) \\
&\sChildren{\vT} = \emptyset \\
&\sPrimes{\vT} = \sPrimes{\sLeft{\vT}} \cup \sPrimes{\sRight{\vT}}
\end{align*}

A komponensek:
\begin{itemize}
\item \sHeight{\vT}: a fa magasságanak felső korlátja,
	ahol egy fa magassága a leghosszabb gyökérből induló út elágazásainak száma.
\item \sPosition{\vT}: a szám, amin a fa áll.
\item \sDiff{\vT}: távolság a "következő" szitált számig
\item \sPrime{\vT}: a levélhez tartozó prím.
\item \sAccumulated{\vT}: a léptetések száma, amennyivel ez a csúcs előrébb jár a gyerekeinél
\item \sLeft{\vT}, \sRight{\vT}: az elágazás bal, és jobb gyereke
\item \sChildren{\vT}: a csúcs gyerekei
\item \sPrimes{\vT}: a fa leveleihez tartozó prímek
\end{itemize}

\section{Műveletek}

A $\fMove(\vT, \vN)$ a fát $\vN$ lépéssel előrébb lépteti a szitatáblán.
A $\fMove$ általában nem végzi el a fa teljes léptetését, csak feljegyzi, hogy később el kell végezni.

A $\fFix(\vT)$ a fát az el nem végzett léptetések részleges elvégzésével olyan állapotba hozza,
hogy a következő szitált szám távolsága az éppen aktuálistól megállapítható.

\begin{align*}
\fFix &: \Trees \rightarrow \Trees \\
\fMove &: \Trees \times \NaturalsP \rightarrow \Trees \\
\end{align*}

\subsection{Levelek}

T.f.h.
\begin{align*}
\vT &\in \Trees_0 \\
\vN &\in \NaturalsP
\end{align*}
Ekkor
\begin{align*}
\fFix(\vT) =& \vT &, \sDiff{\vT} \ge 0 \\
	& \fFix((0, \sPosition{\vT}, \sDiff{\vT} + \sPrime{\vT}, \sPrime{\vT}))
	&, \sDiff{\vT} < 0 \\
\fMove(\vT, \vN)
	=& (0, \sPosition{\vT} + \vN, \sDiff{\vT} - \vN, \sPrime{\vT})& \\
\end{align*}

$\fFix$ rekurziójának szükségessége/helyettesíthetősége $\sDiff{\vT}$ alsó korlátjától függ.

\subsection{Belső csúcsok}

T.f.h.
\begin{align*}
\vH &\in \NaturalsP \\
\vT &\in \Trees_{\vH} \\
\vN &\in \NaturalsP
\end{align*}
Ekkor
\begin{align*}
\fMove(\vT, \vN) =& ( \sHeight{\vT},
	\sPosition{\vT} + \vN,
	\sDiff{\vT} - \vN,
	\sAccumulated{\vT} + \vN,
	\sLeft{\vT},
	\sRight{\vT}) & \\
\vL' \coloneqq& \fMove(\sLeft{\vT}, \sAccumulated{\vT}) \\
\vL'' \coloneqq& \fFix(\vL') \\
\vR' \coloneqq& \fMove(\sRight{\vT}, \sAccumulated{\vT}) \\
\vR'' \coloneqq& \fFix(\vR') \\
\fFix(\vT) =& \vT &, \sDiff{\vT} \ge 0 \\
	=& (\sHeight{\vT}, \sPosition{\vT}, 0, 0, \vL'', \vR')
		&,\sDiff{\vT}<0 \land \sDiff{\vL''} = 0 \\
	=& (\sHeight{\vT}, \sPosition{\vT},
		\min\{\sDiff{\vL''}, \sDiff{\vR''}\},
		0, \vL'', \vR'')
		&, \sDiff{\vT}<0 \land \sDiff{\vL''} > 0 \\
\end{align*}

\section{Invariáns}

\begin{align*}
\fInv :& \Trees \rightarrow \mathbb{L} \\
\fInv(\vT)
	=& \exists \vP \in \sPrimes{\vT}:
		p \mid \sPosition{\vT} + \sDiff{\vT} \tag{I1} \\
	\land & \forall \vP \in \sPrimes{\vT}:
		\forall \vN \in
			\left[\ \sPosition{\vT} , \sPosition{\vT} + \sDiff{\vT}\ \right):
				\vP \nmid \vN \tag{I2} \\
	\land & \forall \vC \in \sChildren{\vT}: \\
		& \qquad \fInv(\vC) \tag{I3} \\
		& \qquad \land \sPosition{\vT} - \sAccumulated{\vT}
			= \sPosition{\vC}  \tag{I4}
\end{align*}

\begin{itemize}
\item I1: $\sPosition{\vT}+\sDiff{\vT}$-et szitálni kell.
\item I2: $\sPosition{\vT}+\sDiff{\vT}$ a "legközelebbi" szám, amit szitálni kell, olyan értelemben, 		hogy minden ennél kisebb szitálandó számot a fa már elhagyott.
	$\sDiff{\vT}<0$ esetén ez triviálisan igaz.
\item I3: elágazás részfái is jók.
\item I4: elágazásban a gyerekfák pontosan $\sAccumulated{\vT}$ lemaradásban vannak.
Ez kiterjeszthető az összes leszármazott részfára az oda vezető út
$ \sum \sAccumulated{\vT} $ lemaradásával.
\end{itemize}

\section{Elő-, és utófeltételek}

T.f.h.
\begin{align*}
\vN \in \NaturalsP \\
\vT \in \Trees \\
\fInv(\vT)
\end{align*}
Ekkor
\begin{align*}
\vT' = \fFix(\vT) \implies
	& \vT' \in \Trees_{\sHeight{\vT}} \tag{F1} \\
	& \land \sPrimes{\vT'} = \sPrimes{\vT} \tag{F2} \\
	& \land \sPosition{\vT'} = \sPosition{\vT} \tag{F3} \\
	& \land \sDiff{\vT'} \ge 0 \tag{F4} \\
	& \land \fInv(\vT') \tag{F5} \\
\vT' = \fMove(\vT, \vN) \implies
	& \vT' \in \Trees_{\sHeight{\vT}} \tag{M1} \\
	& \land \sPrimes{\vT'} = \sPrimes{\vT} \tag{M2} \\
	& \land \sPosition{\vT'} = \sPosition{\vT} + \vN \tag{M3} \\
	& \land \fInv(\vT') \tag{M4}
\end{align*}

\begin{itemize}
\item F1, F2, F3, F5, M1, M2, M4: a fa nem változik lényegesen.
\item F4: a "következő" szitált szám értelmezett, nem lépett túl rajta a fa.
\item M3: a fa új számra lép.
\end{itemize}

Bizonyítás a fa magasságára vonatkozó indukcióval.

\subsection{Levelek}

\subsubsection{$ \fMove(\vT, \vN) $}

T.f.h.
\begin{align*}
\vN \in \NaturalsP \\
\vT \in \Trees_0 \\
\fInv(\vT) \\
\vT' = \fMove(\vT, \vN)
\end{align*}

Ekkor
\begin{align*}
\vT' =& (0, \sPosition{\vT} + \vN, \sDiff{\vT} - \vN, \sPrime{\vT})
	\in \Trees_0 \tag{M1} \\
\sPrimes{\vT'} =& \{ \sPrime{\vT} \} = \sPrimes{\vT} \tag{M2} \\
\sPosition{\vT'} =& \sPosition{\vT} + \vN \tag{M3} \\
\end{align*}
\begin{align*}
\sPrime{\vT} \mid & \sPosition{\vT} + \sDiff{\vT} = \\
	& = \sPosition{\vT} + \vN + \sDiff{\vT} - \vN & \\
	& = \sPosition{\vT'} + \sDiff{\vT'} \tag{I1}
\end{align*}
\begin{align*}
! \vN \in& \left[\ \sPosition{\vT'} , \sPosition{\vT'} + \sDiff{\vT'}\ \right) \\
=& \left[\ \sPosition{\vT} + \vN , \sPosition{\vT} + \vN + \sDiff{\vT} - \vN\ \right) \\
=& \left[\ \sPosition{\vT} + \vN , \sPosition{\vT} + \sDiff{\vT} \right) \\
\subseteq&  \left[\ \sPosition{\vT} , \sPosition{\vT} + \sDiff{\vT} \right) \\
\sPrime{\vT} &\nmid \vN \tag{I2}
\end{align*}
\begin{align*}
\sChildren{\vT'} = \sChildren{\vT} = \emptyset \tag{I3, I4, M4}
\end{align*}

\subsubsection{$ \fFix(\vT), \sDiff{\vT} \ge 0 $ }

$ \sDiff{\vT} $-re vonatkozó indukcióval. T.f.h.
\begin{align*}
\vT \in \Trees_0 \\
\fInv(\vT) \\
\sDiff{\vT} \ge 0 \\
\vT' = \fFix(\vT)
\end{align*}
Ekkor
\begin{align*}
\vT' = \vT \\
\vT' \in \Trees_0 \tag{F1} \\
\sPrimes{\vT'} = \sPrimes{\vT} \tag{F2} \\
\sPosition{\vT'} = \sPosition{\vT} \tag{F3} \\
\sDiff{\vT'} = \sDiff{\vT} \ge 0 \tag{F4} \\
\fInv(\vT') = \fInv(\vT) \tag{F5}
\end{align*}

\subsubsection{$ \fFix(\vT), \sDiff{\vT} < 0 $ }

T.f.h. $ \sDiff{\vT} < 0 $ és az utófeltételeket
$ \forall \vT' \in \Trees_0, \sDiff{\vT'} > \sDiff{\vT} $-re igazoltuk. Ekkor
\begin{align*}
! \vT'' = (0, \sPosition{\vT}, \sDiff{\vT} + \sPrime{\vT}, \sPrime{\vT}) \\
\vT'' \in \Trees_0 \\
\sDiff{\vT''} > \sDiff{\vT} \\
\fInv(\vT'')
\end{align*}
ugyanis
\begin{align*}
\sPrimes{\vT''} & = \left\lbrace \sPrime{\vT} \right\rbrace
	= \sPrimes{\vT} & \\
\sPrime{\vT} & \mid \sPosition{\vT} + \sDiff{\vT} \\
\sPrime{\vT} & \mid \sPosition{\vT''} + \sDiff{\vT''} \\
	& = \sPosition{\vT} + \sDiff{\vT} + \sPrime{\vT} \tag{I1}
\end{align*}
\begin{align*}
& \mathopen| \left[\ \sPosition{\vT''},
	\sPosition{\vT''} + \sDiff{\vT''}\ \right) \mathclose| \\
=& \mathopen| \left[\ \sPosition{\vT},
	\sPosition{\vT} + \sDiff{\vT} +\sPrime{\vT} \ \right) \mathclose| \\
=& \mathopen| \left[\ \sPosition{\vT},
	\sPosition{\vT} + \sDiff{\vT} \ \right]
	\cup \left(\ \sPosition{\vT} + \sDiff{\vT},
	\sPosition{\vT} + \sDiff{\vT} +\sPrime{\vT} \ \right) \mathclose| \\
=& \mathopen| \emptyset \cup \left(\ \sPosition{\vT} + \sDiff{\vT},
	\sPosition{\vT} + \sDiff{\vT} +\sPrime{\vT} \ \right) \mathclose| \\
=& \sPrime{\vT}-1 < \sPrime{\vT} \tag{I2}
\end{align*}
\begin{align*}
\sChildren{\vT''} = \emptyset \tag{I3, I4}
\end{align*}

Definíció szerint
\begin{align*}
\vT' =& \fFix(\vT'')
\end{align*}

Az indukciós feltevés, és $\vT''$ tulajdonságai szerint $ \fFix(\vT'') $
utófeltételei teljesülnek
\begin{align*}
\sHeight{\vT'} =& \sHeight{\vT''} = 0 \tag{F1} \\
\sPrimes{\vT'} =& \sPrimes{\vT''} = \sPrimes{{\vT}} \tag{F2} \\
\sPosition{\vT'} =& \sPosition{\vT''} = \sPosition{\vT} \tag{F3} \\
\sDiff{\vT'} \ge& 0 \tag{F4} \\
\fInv(\vT') \tag{F5}
\end{align*}

\subsection{Belső csúcsok}

Legyen
\begin{align*}
\vH \in& \NaturalsP \\
\vT \in& \Trees_{\vH} \\
\fInv(\vT) \\
\vL' \coloneqq& \fMove(\sLeft{\vT}, \sAccumulated{\vT}) \\
\vL'' \coloneqq& \fFix(\vL') \\
\vR' \coloneqq& \fMove(\sRight{\vT}, \sAccumulated{\vT}) \\
\vR'' \coloneqq& \fFix(\vR') \\
\end{align*}
Mivel
$\max\left\lbrace\ \sHeight{\sLeft{\vT}}, \sHeight{\sRight{\vT}} \right\rbrace
	<\sHeight{\vT}$, $\fInv(\vT)$, így az indukciós feltevés szerint
\begin{align*}
\fInv(\vL') \\
\fInv(\vL'') \\
\fInv(\vR') \\
\fInv(\vR'') \\
\sDiff{\vL''} \ge 0 \\
\sDiff{\vR''} \ge 0
\end{align*}

\subsubsection{$ \fMove(\vT, \vN) $}

\begin{align*}
\vN \in \NaturalsP \\
\vT' = \fMove(\vT, \vN)
\end{align*}

Ekkor
\begin{align*}
\vT' = ( \sHeight{\vT},
	\sPosition{\vT} + \vN,
	\sDiff{\vT} - \vN,
	\sAccumulated{\vT} + \vN,
	\sLeft{\vT},
	\sRight{\vT})
\end{align*}
\begin{align*}
\vT' \in& \Trees_{\vH} \tag{M1} \\
\sPrimes{\vT'} =& \sPrimes{\sLeft{\vT}} \cup \sPrimes{\sRight{\vT}} & \\
	=& \sPrimes{\vT} \tag{M2} \\
\sPosition{\vT'} =& \sPosition{\vT} + \vN \tag{M3} \\
\end{align*}
\begin{align*}
\exists \vP \in \sPrimes{\vT'}: \vP \mid & \sPosition{\vT'} + \sDiff{\vT'} \\
	& = \sPosition{\vT} + \vN + \sDiff{\vT} - \vN \\
	& = \sPosition{\vT} + \sDiff{\vT} \tag{I1}
\end{align*}
\begin{align*}
! \vP \in& \sPrimes{\vT'} \\
! \vN \in& \left[\ \sPosition{\vT'} , \sPosition{\vT'} + \sDiff{\vT'}\ \right) \\
=& \left[\ \sPosition{\vT}+\vN, \sPosition{\vT}+\vN+\sDiff{\vT}-\vN\ \right) \\
=& \left[\ \sPosition{\vT} + \vN , \sPosition{\vT} + \sDiff{\vT} \right) \\
\subseteq&  \left[\ \sPosition{\vT} , \sPosition{\vT} + \sDiff{\vT} \right) \\
\vP \nmid \vN \tag{I2}
\end{align*}
\begin{align*}
! \vC \in \sChildren{\vT'} = \{ \sLeft{\vT}, \sRight{\vT} \} = \sChildren{\vT} \\
\fInv(\vC) \tag{I3}
\end{align*}
\begin{align*}
\sPosition{\vT'} - \sAccumulated{\vT'}
	=& \sPosition{\vT} + \vN - \sAccumulated{\vT} - \vN \\
	=& \sPosition{\vT} - \sAccumulated{\vT} \\
	=& \sPosition{\vC} \tag{I4, M4}
\end{align*}

\subsubsection{$ \fFix(\vT), \sDiff{\vT} \ge 0 $}

$\sDiff{\vT}$-re vonatkozó indukcióval. T.f.h.
\begin{align*}
\vT' = \fFix(\vT)
\end{align*}

Ekkor ha $ \sDiff{\vT} \ge 0 $
\begin{align*}
\vT' = \vT \\
\vT' \in \Trees_{\vH} \tag{F1} \\
\sPrimes{\vT'} = \sPrimes{\vT} \tag{F2} \\
\sPosition{\vT'} = \sPosition{\vT} \tag{F3} \\
\sDiff{\vT'} = \sDiff{\vT} \ge 0 \tag{F4} \\
\fInv(\vT') = \fInv(\vT) \tag{F5}
\end{align*}

\subsubsection{$ \fFix(\vT), \sDiff{\vT} < 0, \sDiff{\vL''} = 0 $}

T.f.h. az indukció szerint már
$\forall \vT' \in \Trees_{\vH}, \sDiff{\vT'}>\sDiff{\vT}$-re
beláttuk $\fFix(\vT)$ utófeltételeit,
$ \sDiff{\vT} < 0 $ és $ \sDiff{\vL''} = 0 $
\begin{align*}
\vT' = (\sHeight{\vT}, \sPosition{\vT}, 0, 0, \vL'', \vR') \\
\vT' \in \Trees_{\vH} \tag{F1} \\
\sPrimes{\vT'} = \sPrimes{\vT} \tag{F2} \\
\sPosition{\vT'} = \sPosition{\vT} \tag{F3} \\
\sDiff{\vT'} = 0 \ge 0 \tag{F4}
\end{align*}
\begin{align*}
\exists \vP \in \sPrimes{\vL''} \subset \sPrimes{\vT'}:
	\vP \mid \sPosition{\vL''} + \sDiff{\vL''} \\
\sPosition{\vL''} + \sDiff{\vL''} =& \sPosition{\vL''} \\
	=& \sPosition{\sLeft{\vT}} + \sAccumulated{\vT} \\
	=& \sPosition{\vT} \\
	=& \sPosition{\vT'} \\
	=& \sPosition{\vT'} + \sDiff{\vT'} \tag{I1}
\end{align*}
\begin{align*}
\left[\ \sPosition{\vT'} , \sPosition{\vT'} + \sDiff{\vT'}\ \right)
	= \emptyset \tag{I2}
\end{align*}
\begin{align*}
\fInv(\vL''), \fInv(\vR') \tag{I3}
\end{align*}
\begin{align*}
\sPosition{\vT'} - \sAccumulated{\vT'}
	=& \sPosition{\vT'} - 0 \\
	=& \sPosition{\vT} \\
	=& \sPosition{\sLeft{\vT}} + \sAccumulated{\vT} \\
	=& \sPosition{\vL''} \\
\sPosition{\vT'} - \sAccumulated{\vT'}
	=& \sPosition{\vT'} - 0 \\
	=& \sPosition{\vT} \\
	=& \sPosition{\sRight{\vT}} + \sAccumulated{\vT} \\
	=& \sPosition{\vR'} \tag{I4, F5}
\end{align*}

\subsubsection{$ \fFix(\vT), \sDiff{\vT} < 0, \sDiff{\vL''} > 0 $}

Ha $ \sDiff{\vT} < 0 $ és $ \sDiff{\vL''} > 0 $
\begin{align*}
\vT' = ( & \sHeight{\vT}, \sPosition{\vT},
	\min\{\sDiff{\vL''}, \sDiff{\vR''}\},
	0, \vL'', \vR'')
\end{align*}
\begin{align*}
\vT' \in& \Trees_{\vH} \tag{F1} \\
\sPrimes{\vT''} =& \sPrimes{\vT} \tag{F2} \\
\sPosition{\vT'} =& \sPosition{\vT} \tag{F3} \\
\sDiff{\vT'} \ge& 0 \tag{F4}
\end{align*}

T.f.h. $ \sDiff{\vT'} = \sDiff{\vL''} $.
A $ \sDiff{\vT'} = \sDiff{\vR''} $ eset szimmetrikus.
\begin{align*}
\exists \vP \in \sPrimes{\vL''} \subset \sPrimes{\vT'}:
	\vP \mid \sPosition{\vL''} + \sDiff{\vL''}
\end{align*}
\begin{align*}
\sPosition{\vL''} + \sDiff{\vL''}
	=& \sPosition{\vL''} + \sDiff{\vT'} \\
	=& \sPosition{\sLeft{\vT}} + \sAccumulated{\vT} + \sDiff{\vT'} \\
	=& \sPosition{\vT} - \sAccumulated{\vT} + \sAccumulated{\vT} + \sDiff{\vT'} \\
	=& \sPosition{\vT} + \sDiff{\vT'} \\
	=& \sPosition{\vT'} + \sDiff{\vT'} \tag{I1}
\end{align*}
\begin{align*}
& \left[\ \sPosition{\vT'} , \sPosition{\vT'} + \sDiff{\vT'}\ \right) \\
= & \left[\ \sPosition{\vT} , \sPosition{\vT} + \sDiff{\vL''}\ \right) \\
= & \left[\ \sPosition{\sLeft{\vT}} + \sAccumulated{\vT},
	\sPosition{\sLeft{\vT}} + \sAccumulated{\vT} + \sDiff{\vL''}\ \right) \\
= & \left[\ \sPosition{\vL''} , \sPosition{\vL''} + \sDiff{\vL''}\ \right)\\
\\
& \left[\ \sPosition{\vT'} , \sPosition{\vT'} + \sDiff{\vT'}\ \right) \\
= & \left[\ \sPosition{\vT} , \sPosition{\vT} + \sDiff{\vL''}\ \right) \\
\subset & \left[\ \sPosition{\vT} , \sPosition{\vT} + \sDiff{\vR''}\ \right) \\
= & \left[\ \sPosition{\sRight{\vT}} + \sAccumulated{\vT},
	\sPosition{\sRight{\vT}} + \sAccumulated{\vT} + \sDiff{\vR''}\ \right) \\
= & \left[\ \sPosition{\vR''} , \sPosition{\vR''} + \sDiff{\vR''}\ \right)
	\tag{I2}
\end{align*}
\begin{align*}
\fInv(\vL''), \fInv(\vR'') \tag{I3}
\end{align*}
\begin{align*}
\sPosition{\vT'} - \sAccumulated{\vT'}
	=& \sPosition{\vT} - 0 \\
	=& \sPosition{\vT} \\
	=& \sPosition{\sLeft{\vT}} + \sAccumulated{\vT} \\
	=& \sPosition{\vL''} \\
\sPosition{\vT'} - \sAccumulated{\vT'}
	=& \sPosition{\vT} - 0 \\
	=& \sPosition{\vT} \\
	=& \sPosition{\sRight{\vT}} + \sAccumulated{\vT} \\
	=& \sPosition{\vR''} \tag{I4, F5}
\end{align*}

\section{Szita algoritmus}

\subsection{Invariáns}

\subsection*{Bizonyítás}

\section{Fa inicializálása}

\subsection{Inicializálás költsége}

A maradékos osztások végrehajtási ideje, ha $ \left[\vU, \vV\right] $ intervallumon szitálunk,
és

\begin{align*}
\sqrt{\vV} \le \vU < \vV \\
\vU \in \bigOrdo(\vV) \\
\mathbb{P}(\vI \text{prím}) \approx \frac{1}{\log{\vI}} \\
\vU \mod \vP \text{költsége} \in \bigOrdo(\log{\vU}\log{\vP})
\end{align*}

akkor

\begin{align*}
\sum_{\substack{\vP=1 \\ \vP \text{prím}}}^{\sqrt{\vV}}
	(\vU \mod \vP) \text{költsége}
	\approx &
	\sum_{\vI=1}^{\sqrt{\vV}}
	\mathbb{P}(\vI \text{prím}) ((\vU \mod \vI) \text{költsége}) \approx \\
\approx& \sum_{\vI=1}^{\sqrt{\vV}}
	\frac{1}{\vC \log{i}} \vD \log{\vU} \log{\vI} = \\
=& \frac{\vD}{\vC} \sqrt{\vV} \log{\vU} \\
\in & \bigOrdo(\sqrt{\vV}\log{\vV})
\end{align*}

Ez több, mint $\sqrt{\vV}$-ig szitálni, ami $\bigOrdo(\sqrt{\vV}\log{\log{\sqrt{\vV}}})$.

Ha az osztás futási ideje (talán Barrett reduction, Karatsuba szorzással)

\begin{align*}
\vU \mod \vP \text{költsége} \in
	\bigOrdo\left(\log^{\log_2{3}}{\vU} + \log{\vU} \log{\log{\vU}} \right)
\end{align*}

akkor

\begin{align*}
\sum_{\substack{\vP=1 \\ \vP \text{prím}}}^{\sqrt{\vV}}
	(\vU \mod \vP) \text{költsége}
	\approx &
	\sum_{\vI=1}^{\sqrt{\vV}}
	\vC \left(\log^{\log_2{3}}{\vU} + \log{\vU} \log{\log{\vU}} \right) \\
	\approx & \frac{\vD\sqrt{\vV}}{\log{\sqrt{\vV}}} \vC \left(\log^{\log_2{3}}{\vU} + \log{\vU} \log{\log{\vU}} \right) \\
	\approx & \frac{\vD\sqrt{\vV}}{\log{\sqrt{\vV}}} \vC \left(\log^{\log_2{3}}{\vV} + \log{\vV} \log{\log{\vV}} \right) \\
	= & \frac{2\vC\vD\sqrt{\vV}}{\log{\vV}} \left(\log^{\log_2{3}}{\vV} + \log{\vV} \log{\log{\vV}} \right) \\
	= & 2\vC\vD\sqrt{\vV} \left(\log^{\log_2{3}-1}{\vV} + \log{\log{\vV}} \right) \\
	\in & \bigOrdo \left( \sqrt{\vV} \left(\log^{\log_2{3}-1}{\vV} + \log{\log{\vV}} \right) \right)
\end{align*}

ami szintén legalább a $\sqrt{\vV}$-ig szitálás ideje.

\section{Fa bővítése}

\section{Kerék}

\subsection{Gyökér}

\subsection{Levelek}

\section{Korlátos alaptípusok}

\begin{itemize}
\item $ \sHeight{\vT} $, és $ \sPosition{\vT} $ tárolása nem szükséges
\item $ ? < \sDiff{\vT} < min\{\sPrimes{\vT}\} $
\item $ 0 \le \sAccumulated{\vT} < ? $
\item $ \max \left\lbrace \sAccumulated{\vT} \right\rbrace
	\approx \max \left\lbrace -\sDiff{\vT} \right\rbrace $ ?
\end{itemize}

T.f.h. algoritmus az aktuális fa gyökerét mindig $\vN = 1$ lépéssel mozgatja.
\begin{align*}
\vT' \coloneqq \fMove(\vT, 1)
\end{align*}
Ekkor a fa $ \forall \vT'' $ csúcsában $ \fMove(\vT'', \vN'') $ műveletnél
\begin{align*}
\vN'' \le& \textrm{\vT''-től balra lévő prímek által generált legnagyobb prímhézag}
\end{align*}
{ \color{red} Bizonyítás? }

A kerék minden csúcstól balra van, és a kerékkel lépve a konkatenált fában egyesével lép az algoritmus. { \color{red} Bizonyítás? }

Módosítsuk a műveleteket úgy, hogy bevezetünk egy korlátot, és egy szűkítést
\begin{align*}
\vP \coloneqq& \max \sPrimes{\vT} \\
\vK \in& \NaturalsP \\
\vK \ge& \vP \\
\Trees_0 =&
	\times \left[ -\vK \mathellipsis \vP-1 \right]
	\times \left[ 1 \mathellipsis \vP \right] \\
\Trees_{\vN+1} =&
	\times \left[ -\vK \mathellipsis \vP-1 \right]
	\times \left[ 0 \mathellipsis \vK-1 \right]
	\times \bigcup_{\vI=0}^{\vN} \Trees_{\vI}
	\times \bigcup_{\vI=0}^{\vN} \Trees_{\vI}
	& (\vN \in \NaturalsZ) \\
\fMove :& \Trees \times \left[ 1 \mathellipsis \vK \right] \rightarrow \Trees
\end{align*}
És a $ \fMove $ műveletet módosítsuk úgy, hogy ha túllépné az alaptípusok korlátait,
akkor végezzen el egy extra $ \fFix $ műveletet.
Az ideiglenes számításokhoz így legfeljebb $ +1 $ bit szükséges a túlcsordulások megállapítására.
{ \color{red} Részletek? Bizonyítás? }

\end{document}

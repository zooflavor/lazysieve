\pagebreak
\section{A program ellenőrzése}

Az megoldás szitáinak ellenőrzése a gui program elvárt feladata.
Az init és generator program szegmensfájlai, és a gui szitáinak kimenetei is összehasonlíthatóak a ReferenceSegment osztály szitájával.
Ezen keresztül a sziták kimenetének egyenlősége szúrópróbaszerűen ellenőrizhető.
Ezeket az időigényes ellenőrzéséket végzi el a ''test-sieves'' szkript minden szitával több intervallumon.
A teszt ellenőrzéséhez a szkript egy szándékosan hibás szegmens, és egy hibás szita ellenőrzését is elvégzi.

A gui program ''math'' és ''util'' csomagjaira a program összes többi része épít, az adatbázisműveletek, a sziták, és a megjelenítés is. Erre a két csomagra a JUnit keretrendszerrel és a JaCoCo lefedettségmérővel teljes lefedettségű egységtesztek készültek.

Az összesített statisztikák ellenőrzésére a ''test-prime-counts'' szkript a prímek számát a ''samples/test/prime-counts.csv'' fájlba írja az adatbázis alapján kettő hatványai helyeken, míg a ''test-max-prime-gaps'' a ''samples/test/max-prime-gaps.csv'' fájlba a maximális prímhézagokat metni.
Ezek a publikált értékekkel\cite{gaps}\cite{pi} összevethetőek. Ezeket a fájlokat a gui program ''gui.test'' csomagjának osztályai készítik el. A program tesztelésénél ez a két statisztika $2^{47}(\approx1,4\cdot10^{14})$-ig egyezett.

A minták és közelítő függvények megjelenítésének tesztelése ismert grafikonú mintákkal való összevetéssel történt.
Táblázatkezelő programokkal, mint például a LibreOffice, formula alapján generált minták grafikonja ábrázolható, és a gui program által olvasott CSV formátu fájlba menthető.
A formulákkal véletlenszerű zaj is hozzáadható a mintához.
A ''samples/test/graph'' könyvtárban megtalálható néhány táblázatminta a teszteléshez.

\subsection{Numerikus pontosság}

A legkisebb négyzetek módszerének megvalósítása a mátrixműveletek pontosságát két paraméterrel tudja szabályozni, a mátrixszorzás összegzéseit három különböző algoritmussal lehet végezni, és a Gauss-eliminációt részleges sorcserével, vagy teljes sor-oszlop cserével is el lehet végezni.
Ezeknek a paramétereknek a megválasztása a tesztek alapján sem egyértelmű.

A ''test-measure-sums'' közelítések pontosságáról készít statisztikákat három változóban, és az eredményt a ''samples/test/measure-sums.csv'' fájlba menti. A három váltózó:
\begin{itemize}
\item az összegzést tömbbel, kupaccal, vagy egyetlen változóval végezze
\item részleges vagy teljes csere
\item x szerint rendezett-e a minta.
\end{itemize}
A három paraméter mindegyik lehetőségéhez a mérés elemi függvények alapján egy mintát generál, majd ezt a mintát ugyanezekkel az elemi függvényekkel közelíti.
Egy paraméter-kombinációhoz az összes közelítés négyzetes eltérését, és a közelítés idejét összesíti.
A minták generálásához használt elemi függvények a ''MeasureSums'' osztályból kiolvashatóak.

A mérés eredményéből látszik, hogy a kiválasztott minták közelítésénél:
\begin{itemize}
\item legjobb, és leglassabb választás a kupac összegzés teljes cserével
\item a kupac összegzés tízszer annyi időt vesz igénybe, mint a másik két összegzés
\item az egy változóban összegzés hibája több nagyságrenddel nagyobb, mint a másik két összegzés hibája
\item a sor és oszlop csere nem számít egy változóban összegezve, a teljes csere tömbben valamivel pontosabb, és a kupac összegzésnél a teljes csere lényegesen pontosabb eredmény ad
\item a kupac összegzés a minta rendezettségére nem érzékeny
\item a rendezetlen minta összegzése a változóval és tömbbel pontosabb, mint a rendezett.
\end{itemize}
Ezek alapján a program a közelítésekre a kupac alapú összegzést használja, és a Gauss-eliminációban teljes sor és oszlopcserét hajt végre, ezt a felhasználó nem tudja megváltoztatni.

\begin{table}[h]
\renewcommand\arraystretch{1.2}
\centering
\caption{Összegző algoritmusok összehasonlítása}
\begin{tabular}{|l|l|l|l|l|}
\hline
\bf{Összegzés} & \bf{Csere} & \bf{Rendezett} & \bf{$\sum{\textrm{hiba}^2}$} & \bf{Idő (ns)} \\ \cline{1-5}
\multirow{4}{*}{változó} & \multirow{2}{*}{részleges} & igen & \num{2,53e25} & \num{1,26e9} \\ \cline{3-5}
& & nem & \num{4,24e23} & \num{1,85e9} \\ \cline{2-5}
& \multirow{2}{*}{teljes} & igen & \num{2,53e25} & \num{1,34e9} \\ \cline{3-5}
& & nem & \num{4,21e23} & \num{1,82e9} \\ \cline{1-5}
\multirow{4}{*}{tömb} & \multirow{2}{*}{részleges} & igen & \num{2,62e20} & \num{1,45e9} \\ \cline{3-5}
& & nem & \num{9,75e19} & \num{2,45e9} \\ \cline{2-5}
& \multirow{2}{*}{teljes} & igen & \num{1,87e20} & \num{1,56e9} \\ \cline{3-5}
& & nem & \num{6,76e19} & \num{2,33e9} \\ \cline{1-5}
\multirow{4}{*}{kupac} & \multirow{2}{*}{részleges} & igen & \num{1,12e21} & \num{1,37e10} \\ \cline{3-5}
& & nem & \num{1,12e21} & \num{1,67e10} \\ \cline{2-5}
& \multirow{2}{*}{teljes} & igen & \num{3,12e19} & \num{1,39e10}  \\ \cline{3-5}
& & nem & \num{3,12e19} & \num{1,68e10} \\
\hline
\end{tabular}
\end{table}

%%% Local Variables:
%%% mode: latex
%%% TeX-master: "szakdolgozat"
%%% End:

\pagebreak
\section{A program ellenőrzése}

A megoldás szitáinak ellenőrzése a \texttt{gui} program feladata.
Az \texttt{init} és \texttt{generator} program szegmensfájlai, és a \texttt{gui} szitáinak kimenetei is a \texttt{ReferenceSegment} osztály szitájával összehasonlíthatóak.
Ezen keresztül a sziták kimenetének egyenlősége szúrópróbaszerűen ellenőrizhető.
Ezeket az időigényes ellenőrzéséket végzi el a \texttt{test-sieves} szkript minden szitával több intervallumon.
A teszt ellenőrzéséhez a szkript egy szándékosan hibás szegmens, és egy hibás szita ellenőrzését is elvégzi.

A \texttt{gui} program \texttt{gui.math} és \texttt{gui.util} csomagjaira a program összes többi része épít, az adatbázis műveletek, a sziták, és a megjelenítés is. Erre a két csomagra a \mbox{JUnit} keretrendszerrel és a JaCoCo lefedettségmérővel teljes lefedettségű egységtesztek készültek.

Az összesített statisztikák ellenőrzésére a \texttt{test-prime-counts} szkript a prímek számát a \texttt{samples/test/prime-counts.csv} fájlba írja az adatbázis alapján kettő hatványai helyeken, míg a \texttt{test-max-prime-gaps} szkript a maximális prímhézagokat a \texttt{samples/test/max-prime-gaps.csv} fájlba menti.
Ezek a publikált értékekkel\cite{gaps}\cite{pi} összevethetőek.
Ezeket a fájlokat a \texttt{gui} program \texttt{gui.test} csomagjának osztályai készítik el.
A program tesztelésénél ez a két statisztika $2^{47}(\approx \num{1,4e14})$-ig egyezett.

A minták és közelítő függvények megjelenítésének tesztelése ismert grafikonú mintákkal való összevetéssel történt.
Táblázatkezelő programokkal, mint például a LibreOffice, formula alapján generált minták grafikonja ábrázolható, és a \texttt{gui} program által olvasott CSV formátumú fájlba menthető.
A formulákkal véletlenszerű zaj is hozzáadható a mintához.
A \texttt{samples/test/graph} könyvtárban megtalálható néhány táblázatminta a teszteléshez.

\subsection{Numerikus pontosság}

A legkisebb négyzetek módszerének megvalósítása a mátrixműveletek pontosságát két paraméterrel tudja szabályozni, a mátrixszorzás összegzéseit három különböző algoritmussal lehet végezni, és a lineáris egyenletrendszereket meg lehet oldani QR felbontással, és Gauss-eliminációval részleges sorcserével és teljes sor és oszlop cserével.
Ezeknek a paramétereknek a megválasztása a tesztek alapján sem egyértelmű.

A \texttt{test-measure-sums} közelítések pontosságáról készít statisztikákat három változóban, és az eredményt a \texttt{samples/test/measure-sums.csv} fájlba menti.
A három váltózó:
\begin{itemize}
\item az összegzést tömbbel, kupaccal, vagy egyetlen változóval végezze
\item az egyenletrendszerek megoldását QR felbontással, részleges vagy teljes Gauss-eliminációval végezze
\item $x$ szerint rendezett-e a minta.
\end{itemize}
A három paraméter mindegyik lehetséges értékéhez a mérés elemi függvények alapján egy mintát generál, majd ezt a mintát ugyanezekkel az elemi függvényekkel közelíti.
Egy paraméter-kombinációhoz az összes közelítés négyzetes eltérését, és a közelítés idejét összesíti.
A minták generálásához használt elemi függvények a \texttt{MeasureSums} osztályból kiolvashatóak.

A mérés eredményéből látszik, hogy a kiválasztott minták közelítésénél:
\begin{itemize}
\item a legkisebb hibát okozó választás a kupac összegzés teljes cserét végrehajtó Gauss-eliminációval
\item a kupac összegzés tízszer annyi időt vesz igénybe, mint a másik két összegzés
\item az egy változóban összegzés hibája több nagyságrenddel nagyobb, mint a másik két összegzés hibája
\item a sor és oszlop csere nem számít egy változóban összegezve, a teljes csere tömbben összegezve valamivel pontosabb, és a kupac összegzésnél a teljes csere lényegesen pontosabb eredmény ad
\item a QR felbontás általában nagyobb hibát eredményez, mint a Gauss-elimináció
\item a kupac összegzés a minta rendezettségére nem érzékeny
\item a rendezetlen minta összegzése a változóval és tömbbel pontosabb, mint a rendezett.
\end{itemize}
Ezek alapján a program a közelítésekre a kupac alapú összegzést használja, és a lineáris egyenletrendszerek megoldására Gauss-eliminációt használ teljes sor és oszlopcserével, ezt a felhasználó nem tudja megváltoztatni.

\begin{table}[H]
\renewcommand\arraystretch{1.2}
\centering
\caption{Numerikus algoritmusok összehasonlítása}
\begin{tabular}{|l|l|l|l|l|}
\hline
\bf{Összegzés} & \bf{LER-m.o.} & \bf{Rendezett} & \bf{$\sum{\textrm{hiba}^2}$} & \bf{Idő (ns)} \\ \cline{1-5}

\multirow{6}{*}{változó} & \multirow{2}{*}{részleges G.E.} & igen & \num{2,53e25} & \num{1,27e9} \\ \cline{3-5}
& & nem & \num{4,24e23} & \num{1,83e9} \\ \cline{2-5}
& \multirow{2}{*}{teljes G.E.} & igen & \num{2,53e25} & \num{1,28e9} \\ \cline{3-5}
& & nem & \num{4,21e23} & \num{1,84e9} \\ \cline{2-5}
& \multirow{2}{*}{QR} & igen & \num{9,47e24} & \num{1,34e9} \\ \cline{3-5}
& & nem & \num{2,58e23} & \num{1,83e9} \\ \cline{1-5}

\multirow{6}{*}{tömb} & \multirow{2}{*}{részleges G.E.} & igen & \num{2,62e20} & \num{1,46e9} \\ \cline{3-5}
& & nem & \num{9,75e19} & \num{2,43e9} \\ \cline{2-5}
& \multirow{2}{*}{teljes G.E.} & igen & \num{1,87e20} & \num{1,53e9} \\ \cline{3-5}
& & nem & \num{6,76e19} & \num{2,37e9} \\ \cline{2-5}
& \multirow{2}{*}{QR} & igen & \num{6,41e23} & \num{1,61e9} \\ \cline{3-5}
& & nem & \num{3,63e20} & \num{2,43e9} \\ \cline{1-5}

\multirow{6}{*}{kupac} & \multirow{2}{*}{részleges G.E.} & igen & \num{1,12e21} & \num{1,40e10} \\ \cline{3-5}
& & nem & \num{1,12e21} & \num{1,76e10} \\ \cline{2-5}
& \multirow{2}{*}{teljes G.E.} & igen & \num{3,12e19} & \num{1,39e10}  \\ \cline{3-5}
& & nem & \num{3,12e19} & \num{1,77e10} \\ \cline{2-5}
& \multirow{2}{*}{QR} & igen & \num{1,30e22} & \num{1,44e10} \\ \cline{3-5}
& & nem & \num{1,30e22} & \num{1,75e10} \\ \cline{1-5}

\hline
\end{tabular}
\end{table}

%%% Local Variables:
%%% mode: latex
%%% TeX-master: "szakdolgozat"
%%% End:

\documentclass[12pt]{report}
\usepackage[a4paper,
			inner = 35mm,
			outer = 25mm,
			top = 25mm,
			bottom = 25mm]{geometry}
\usepackage{lmodern}
\usepackage[magyar]{babel}
\usepackage[utf8]{inputenc}
\usepackage[T1]{fontenc}
\usepackage[unicode]{hyperref}
\usepackage{graphicx}
\usepackage{amssymb}
\usepackage{amsmath}
\usepackage{epstopdf}
\usepackage{setspace}
\usepackage[nottoc,numbib]{tocbibind}
\usepackage{color}
\setcounter{secnumdepth}{3}
\usepackage{algpseudocode}
\usepackage{titlesec}
\onehalfspacing
\begin{document}
\begin{titlepage}
\vspace*{0cm}
\centering
\begin{tabular}{cp{2cm}c}
\begin{minipage}{4cm}
\vspace{0pt}
\includegraphics[width=1\textwidth]{eltecimerszines}
\end{minipage} & &
\begin{minipage}{7cm}
\vspace{0pt}Eötvös Loránd Tudományegyetem \vspace{10pt} \newline
Informatikai Kar \vspace{10pt} \newline
Komputeralgebra Tanszék
\end{minipage}
\end{tabular}

\vspace*{0.2cm}
\rule{\textwidth}{1pt}

\vspace*{6cm}
{\Huge Prímszita algoritmusok összehasonlítása}

\vspace*{5cm}
\begin{tabular}{lp{3cm}l}
Vatai Emil & & Nagy Péter\\
Adjunktus & & Programtervező Informatikus BSc
\end{tabular}

\vfill

\vspace*{1cm}
Budapest, 2018
\end{titlepage}

\tableofcontents

\chapter{Bevezetés}
\section{Motiváció}
\paragraph{x}

\section{Dolgozat eredményei}
\paragraph{x}

\section{Dolgozat szerkezete}
\paragraph{x}

\chapter{Felhasználói dokumentáció}

\chapter{Fejlesztői dokumentáció}

\section{x}

\section{Prioritásos sorok}

\begin{algorithmic}[1]
\State $q \gets$ \Call{Új-sor}{}
\For{$i\gets 2, n$}
	\While{$\exists (p, k) \in q: k \le i$}
		\State $(p, k) \gets $ \Call{Sor-Eltávolít-Min}{q}
		\State \Call{Megjelöl}{i}
		\State \Call{Sor-Beszúr}{q, (p, k+p)}
	\EndWhile
	\If{$\neg$ \Call{Megjelölt?}{i}}
		\State \Call{Sor-Beszúr}{q, (i, 2i)}
	\EndIf
\EndFor
\end{algorithmic}

\subsection{Bináris kupac}

A mérések grafikonják pixelei alapján lassú. A beszúrásonkénti elméleti
$\mathcal{O}(log_{}{|q|})$ ideje se biztató.

\subsection{Bigyó}

A bigyó egy természetesszám-párokat tartalmazó monoton prioritásos sor.
A számpárok egy prímet, és a prím egy szitálási pozícióját reprezentálják.
A sor monoton, minden állapotához tartozik egy érték, a sor aktuális pozíciója,
aminél kisebb vagy egyenlő pozíciójú értéket a sor nem tartalmazhat.
A bigyó edények egy végtelen sorozatát is tárolja, a sor elemei ezekbe az edényekbe kerülnek.
Egy eltárolt elem helyét a sorozatban az elem pozíciójának
és a sor aktuális pozíciójának távolsága határozza meg.

A távolságfüggvény legyen

\begin{align*}
d(x, y) := \lfloor log_{2}{}(x \oplus y) \rfloor & & (x, y \in \mathbb{N}, y > x \ge 0)
\end{align*}

ahol $\oplus$ a bitenkénti XOR.

$d(x, y)$ a legnagyobb bit-index, ahol $x$ és $y$ eltér.

Ha $q$ egy bigyó, legyen $q.a$ $q$ aktuális pozíciója, és $q.e[i]$ $q$ $i$. edénye. AZ edények, és a számpárok struktúrája...

Egy q bigyó invariánsa
\begin{align*}
\forall (p, k) \in q &: &\\ 
	& q.a < k\\
	& \forall i \in \mathbb{N}_{0}: (p, k) \in q.e[i] \iff i=d(q.a, k) \\
\forall (p, k) \not\in q &: \forall i \in \mathbb{N}_{0}: (p, k) \not\in q.e[i]
\end{align*}

Új, üres sor létrehozása tetszőleges kezdőpozíciótól, és meglévő sorba elem beszúrása...

A sor elemeinek feldolgozása $i$-ig
\begin{algorithmic}[1]
\While{$q.a < i$}
	\State $j \gets d(q.a, q.a + 1)$
	\State $q.a \gets q.a + 1$
	\ForAll{$(p, k) \in q.e[j]$}
		\State \Call{Edény-Kivesz}{$q.e[j], (p, k)$}
		\If{$k = i$}
			\State \Call{Visszaad}{$(p, k)$}
		\Else
			\State \Call{Edény-Beszúr}{$d(q.a, k), (p, k)$}
		\EndIf
	\EndFor
\EndWhile
\end{algorithmic}

\subsubsection{Helyesség}

\subsubsection{Idő}

\subsubsection{Hely}

\subsubsection{Számrendszer}

\subsubsection{Cache}

\section{x}

\end{document}